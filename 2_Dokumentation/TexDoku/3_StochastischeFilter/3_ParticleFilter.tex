\section{Partikel-Filter}
Die Eigenschaft des Histogramm-Filters, den Zustandsraum in gleichgroße Zellen zu unterteilen, entpuppt sich bei der Lokalisierung als Nachteil. In den meisten Anwendungsfällen kann ein Großteil des Zustandsraumes recht rasch ausgeschlossen werden, was sich in Form von verschwindend geringen Wahrscheinlichkeiten manifestiert. Es entsteht der Konflikt, dass einerseits eine möglichst grobe Diskretisierung gewählt werden möchte, um keine Ressourcen für irrelevante Gebiete des Zustandsraumes zu verschwenden, andererseits wird durch ein feines Raster die Präzision der Lokalisierung gesteigert. Im den Fall, dass a priori über die Umgebung vorliegen, kann eine ungleichmäßige Diskretisierung des Zustandsraums vorgenommen werden. Allerdings kann weder die Annahme getroffen werden, dass derartige Information über die Verteilung bekannt sind, noch wird dadurch das zu Grunde liegend Problem gelöst. Das Histogramm-Filter ist nicht in der Lage, die Ressourcen, welche die Verteilung repräsentieren, an die aktuellen Wahrscheinlichkeitswerte anzupassen. 

Eine Lösung für dieses Problem liefert das Partikel-Filter, das die Verteilung durch eine finite Menge von Stichproben repräsentiert. Die Proben der Verteilung werden als Partikel bezeichnet, woher auch der Name des Filters stammt. Da die Partikel aus der Verteilung gezogen werden, folgt im Umkehrschluss, dass eine von vielen Partikeln bevölkerte Region mit einer hohen Wahrscheinlichkeit korrespondiert.

\begin{lstlisting}[mathescape=true, caption={Partikel-Filter}, captionpos=bot]
ParticleFilter($\bs{\mathcal{X}}\idx{t-1}$, $\mVec{u}\idx{t}$, $\mVec{z}\idx{t}$)
	$\overline{\bs{\mathcal{X}}}\idx{t} = \bs{\mathcal{X}}\idx{t} = \emptyset$
	for $m = 1$ to $M$ do
		sample $\mVec{x}\idx{t}^{[m]} \leftarrow \condP{\mVec{u}\idx{t}}{\mVec{x}\idx{t}, \mVec{x}\idx{t-1}^{[m]}}$
		$\omega\idx{t}^{[m]} = \condP{\mVec{z}\idx{t}}{\mVec{x}\idx{t}^{[m]}}$
		add $\left(\mVec{x}\idx{t}^{[m]}, \omega\idx{t}^{[m]}\right)$ to $\overline{\bs{\mathcal{X}}}\idx{t}$
	endfor
	for $m = 1$ to $M$ do
		draw $i$ with probability $\omega\idx{t}^{[i]}$ //sollte das nicht folgende Zeile sein
		draw $i$ with probability $\omega\idx{t}^{[m]}$
		add $\mVec{x}\idx{t}^{[i]}$ to $\bs{\mathcal{X}}\idx{t}$
	endfor
	return $\bs{\mathcal{X}\idx{t}}$
\end{lstlisting}
Im ersten Schritt wird der Algorithmus heuristisch erläutert, um ein anschauliches aber dafür oberflächliches Verständnis für dessen Funktionsprinzip zu erhalten. Das Filter arbeitet rekursiv, so wird ihm die Partikelmenge des vorherigen Abtastzeitpunktes $\bs{\mathcal{X}}\idx{t-1}$, sowie der Stellvektor $\mVec{u}\idx{t}$ und Messvektor $\mVec{z}\idx{t}$ übergeben. Im ersten Schritt wird die vorläufige Partikelmenge $\overline{\bs{\mathcal{X}}}\idx{t}$ bestimmt, indem aus der Wahrscheinlichkeit $\condP{\mVec{x}\idx{t}}{\mVec{u}\idx{t}, \mVec{x}\idx{t-1}^{[m]}}$ und den vorherigen Partikeln $\mVec{x}\idx{t-1}^{[m]}$ neue Werte $\mVec{x}\idx{t}^{[m]}$ gezogen werden. Mit der Verteilung liegt ein Bewegungsmodell vor, weshalb die Stichprobe $\mVec{x}\idx{t}^{[m]}$ als der am wahrscheinlichsten auf den alten Zustand $\mVec{x}\idx{t-1}^{[m]}$ und Stellgröße $\mVec{u}\idx{t}$ folgende Zustand interpretiert werden kann. Im nächsten Schritt wird bestimmt, wie wahrscheinlich der Messvektor $\mVec{z}\idx{t}$ unter diesem neuen Partikeln $\mVec{x}\idx{t}^{[m]}$ auftritt. Diese Wahrscheinlichkeit $\omega\idx{t}^{[m]}$ wird auch als Gewichtung bezeichnet. Das so bestimmte Paar $\left(\mVec{x}\idx{t}^{[m]}, \omega\idx{t}^{[m]}\right)$ wird der Menge $\overline{\bs{\mathcal{X}}}\idx{t}$ hinzugefügt. 

Der bisherige Schritt bestand lediglich darin, die Partikel des vorherigen Abtastpunktes der zeitlichen Veränderung anzupassen, woraus die vorläufige Menge $\overline{\bs{\mathcal{X}}}\idx{t}$ resultiert ist. Im Anschluss werden die Partikel mit einer zu der Gewichtung $\omega\idx{t}^{[m]}$ proportionalen Wahrscheinlichkeit neu gezogen. Dieser Schritt wird dadurch motiviert, dass es sich bei $\omega\idx{t}^{[m]}$ um die Wahrscheinlichkeit handelt, dass der Messvektor aus dem Partikel $\mVec{x}\idx{t}^{[m]}$ resultiert. Insofern plausibilisiert $\omega\idx{t}^{[m]}$ sein zugehöriges Partikel $\mVec{x}\idx{t}^{[m]}$. Angenommen das Bewegungsmodell $\condP{\mVec{u}\idx{t}}{\mVec{x}\idx{t}, \mVec{x}\idx{t-1}^{[m]}}$ bringt ein Partikel $\bs{\mathcal{X}}\idx{t}$ hervor, das der Position des Roboters in einer Ecke entspricht. Befindet sich der Roboter in der Realität aber vor einer gerade verlaufenden Wand, so ergibt das Messmodell $\condP{\mVec{z}\idx{t}}{\mVec{x}\idx{t}^{[m]}}$ für das Partikel $\mVec{x}\idx{t}^{[m]}$ eine sehr geringe Gewichtung $\omega\idx{t}^{[m]}$. Weshalb das Partikel mit einer nur sehr geringen Wahrscheinlichkeit in die neue Menge $\bs{\mathcal{X}}\idx{t}$ übernommen werden. Genau durch diesen Mechanismus kompensiert das Partikel-Filter die Nachteile des Histogramm-Filters. Die Plausibilisierung mittels $\omega\idx{t}^{[m]}$ verhindert, das unwahrscheinliche Hypothesen in Form von Partikeln verfolgt werden. Langfristig fokussieren sich die Partikel und somit auch die Ressourcen des Algorithmus auf die wahrscheinlichsten Hypothesen.

{\color{red} Schlechte Erklärung/Formulierung}