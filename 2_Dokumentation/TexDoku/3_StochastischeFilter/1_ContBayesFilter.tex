\section{Kontinuierliches Bayes Filter}
In den späteren Anwendungsfallen werden stochastische Filter primär genutzt, um eine Schätzung der aktuellen Lage des Roboters zu treffen, wofür stochastische Filtermethoden genutzt werden. Die Grundlage bildet das so genannte Bayes-Filter für kontinuierlich verteilte Zustände. Als Beispiel für die Herleitung wird die Lokalisierungsaufgabe herangezogen, bei welcher die aktuelle Position $\mVec{x}\idx{t}$ des Roboters ermittelt werden soll. Die Schätzung stützt sich dabei auf die bisher gesammelten Messwerte $\mVec{z}\idx{1:t}$ und Stellgrößen $\mVec{u}\idx{1:t}$. Insofern kann die Lokalisierungsaufgabe als die Bestimmung der bedingten Wahrscheinlichkeit
\begin{equation}
\condP{\mVec{x}\idx{t}}{\mVec{z}\idx{1:t},\mVec{u}\idx{1:t}}
\end{equation}
reformuliert werden, die im Folgenden auch mittels der Definition
\begin{equation}
\bel{\mVec{x}\idx{t}} \equiv \condP{\mVec{x}\idx{t}}{\mVec{z}\idx{1:t},\mVec{u}\idx{1:t}}
\end{equation}
abgekürzt wird. Im Falle, dass der aktuelle Messwert $\mVec{z}\idx{t}$ bei der Verteilung nicht beachtet wird, gilt
\begin{equation}
\pbel{\mVec{x}\idx{t}} \equiv \condP{\mVec{x}\idx{t}}{\mVec{z}\idx{1:t-1},\mVec{u}\idx{1:t}}\,.
\end{equation}