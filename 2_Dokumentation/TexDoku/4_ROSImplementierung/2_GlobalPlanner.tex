\section{Globaler Planer}
Für die Umsetzung der globalen Plannung wird das ROS-Paket \lstinline{global_planner} \cite{WikiGlobalPlanner} verwendet. Die Funktion der Aktionsgewichtung $f(\mVec{x}\idx{t}, \mVec{u}\idx{t})$ wird von einer Kostenkarte erfüllt, wobei es sich um ein Occupancy-Grid handelt, also einer ortsdiskreten Gitterdarstellung. Die Kostenkarte ordnet jeder Zelle einen Aufwand zu, der aufgebracht werden muss, um die Zelle zu passieren. In dem ROS-Paket kann konfiguriert werden, ob zur Lösung des diskreten Planungsproblems Dijkstras Algorithmus oder $A*$ verwendet werden soll. 

Insgesamt können bei dem Planer sieben Paramter konfiguriert werden \cite{global_planner}, wobei lediglich die folgenden für den Anwenungsfall von Interesse sind:
\newline

\lstinline{allow_unkown (bool, default: true)}{}

Mit dem Paramter wird bestimmt, ob auch Pfade durch Zellen geplant werden soll, bei deren Zustand nicht bekannt ist.
\newline

\lstinline{default_tolerance (double, default: 0.0)}{}

Toleranzabstand mit dem der geplante Pfad die Zielposition erreichen soll. Da bei der Planung eine diskrete Darstellung verwendet wird, entsteht durch die Default-Toleranz von 0 kein Problem.
\newline

\lstinline{use_dijkstra (bool, default:true)}{}

Flag ob Dijkstra oder $A*$ verwendet werden soll. Der Grund für die Standardeinstellung ist, dass $A*$ nach \cite{WikiGlobalPlanner} auf suboptimale Ergebnisse führen kann. Eine mögliche Erklärung für dieses Verhalten besteht darin, dass es sich bei der Heuristik für die Kostenrechnung um keine Unterschätzung des wirklichen Aufwandes handelt.
\newline

\lstinline{use_quadratic (bool, default:true)}{}

Bei diesem Parameter kommt der Internetcharakter der ROS-Dokumentation das erste Mal zum Tragen. Nach \cite{WikiGlobalPlanner} führt das Flag dazu, dass eine quadratische Approximation des Potentials - synonym für die Kosten - genutzt wird. Im Folgesatz schreiben die Autoren dennoch: "Of what, the maintainer of this package has no idea" \cite{WikiGlobalPlanner}. Nach weiterer Internetquelle .. {\color{red} Wie der Paramter genau funktioniert ist hier nicht wirklich relevant, eigentlich sollte man erst darauf eingehen, wenn das wichtig wird}
\newline

\lstinline{use_grid_path (bool, default:false)}{}

Wenn auf \lstinline{true}{} gestellt, folgt der Pfad wie in den obigen Beispielen den Zellen, was dazu führt, dass der Roboter sich nur in acht Richtungen bewegen kann - durch \lstinline{ues_quadratic}{} wird die schräge Bewegung ermöglicht. Der Standardwert des Parameters führt zu einer Glättung des Pfades, woraus rundere Bahnen resultieren.

