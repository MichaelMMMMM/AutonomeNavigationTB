\section{Funktionsprinzip des globalen Planers}
Wie bereits erwähnt, wird die Navigationsaufgabe mithilfe zweier sukzessiver Planungsalgorithmen gelöst. Im ersten Schritt berechnet ein globaler Planer einen Weg zwischen der Ausgangs- und Zielposition, wofür die vorgegebene Karte herangezogen wird. Als Planungsalgorithmus werden im Navigation-Stack entweder Dijkstra- oder der A*-Algorithmus verwendet, die an dieser Stelle näher untersucht werden. Prinzipiell besteht die Aufgabe darin, einen Weg und Trajektorie zu ermitteln, die den Ausgangszustand des Roboters mit dem gewünschten Endzustand verbinden. Bei den gesuchten Pfaden handelt es sich um zeit- und ortskontinuierliche Kurven, woraus ein kontinuierlicher Raum von Lösungen resultiert, der nach einer optimalen durchsucht werden muss. Die Kontinuität des Suchraums stellt auf der Planungsseite eine immense Herausforderung dar, da eine unendliche Zahl von Möglichkeiten zur Verfügung stehen. {\color{red} Viel zu schwammige Erklärung, kleinen Einschub über Probleme wie differenzialgleichungen und Gütefunktionale...} Um dieser Komplexität Herr zu werden wird der Suchraum diskretisiert und die Planungsaufgabe auf die Bestimmung einer Positionsfolge reduziert.

Allerdings wird die Diskretisierung des Raumes von mehr als einer reinen Vereinfachung des Suchproblems motiviert: Der globale Planer übernimmt lediglich den ersten Schritt bei der Berechnung der letztendlichen Trajektorie. Der lokale Planer berechnet anhand der globalen Sollkurve eine lokale Trajektorie, die in Form von Stellgrößen realisiert wird. Dabei werden neben der globalen Ortskurve auch aktuelle Sensor- und Lokalisierungsinformationen miteinbezogen, welche unter Umständen dazu führen können, dass der Roboter von der global geplanten Kurve abweicht. Insofern ergibt es wenig Sinn, bei der globalen Planung Ressourcen in eine unnötig genaue Trajektorie zu investieren; besonders dann, wenn noch nicht alle nötigen Informationen vorliegen. Für den Anfang genügt ein ungenauer, fehlerbehafteter Pfad, der die Anfangs- und Zielposition verbindet, wofür eine diskrete Darstellung des Suchraums vollkommen genügt. Des Weiteren stimmt die Forderung nach einer diskreten Darstellung des Suchraums mit den üblichen Darstellungsformen von metrischen Karten überein, denn diese repräsentieren die Umgebung in Form von diskreten Zellen.

Im Rahmen dieser Überlegung werden die vereinfachten Bedingungen zunächst in einer mathematischen Darstellung formuliert. Es wird angenommen, dass der Roboter in einer planen Umgebung manövriert, welche mittels einer zweidimensionalen Karte dargestellt werden kann. Hierfür wird eine metrische Karte verwendet, die den Raum in quadratische Zellen fixer Größe unterteilt. Die Positionen von Objekten innerhalb der Karten werden mithilfe des X- und Y-Indexes angegeben, wodurch die Positionsangaben von der Zellengröße entkoppelt werden.