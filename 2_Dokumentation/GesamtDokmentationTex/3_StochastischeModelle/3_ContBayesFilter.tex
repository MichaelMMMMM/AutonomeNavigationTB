\section{Kontinuierliches Bayes Filter}
In der Robotik werden stochastische Filter genutzt, um anhand von gegebenen Information wie Mess- und Stellgrößen eine Schätzung über unbekannte Zustände des Systems zu treffen. Die Grundlage für derartige Methoden stellt das Bayes-Filter für kontinuierlich verteilte Zustände dar. Als Beispiel für die Herleitung und Illustration des Filters wird die Lokalisierungsaufgabe herangezogen, bei welcher die aktuelle Position $\mVec{x}\idx{t}$ des Roboters ermittelt werden soll. Die Schätzung stützt sich dabei auf die bisher gesammelten Messwerte $\mVec{z}\idx{1:t}$ und Stellgrößen $\mVec{u}\idx{1:t}$. Insofern kann die Lokalisierungsaufgabe als die Bestimmung der bedingten Wahrscheinlichkeit
\begin{equation}
\condP{\mVec{x}\idx{t}}{\mVec{z}\idx{1:t},\mVec{u}\idx{1:t}}
\end{equation}
reformuliert werden, die im Folgenden auch mittels der Definition
\begin{equation}
\bel{\mVec{x}\idx{t}} \equiv \condP{\mVec{x}\idx{t}}{\mVec{z}\idx{1:t},\mVec{u}\idx{1:t}}
\end{equation}
abgekürzt wird. Im Falle, dass der aktuelle Messwert $\mVec{z}\idx{t}$ bei der Verteilung nicht beachtet wird, gilt
\begin{equation}
\pbel{\mVec{x}\idx{t}} \equiv \condP{\mVec{x}\idx{t}}{\mVec{z}\idx{1:t-1},\mVec{u}\idx{1:t}}\,.
\end{equation}
Für die Bestimmung der Filtergleichungen wird auf Bayes Theorem zurückgegriffen:
\begin{equation}
\begin{split}
\bel{\mVec{x}\idx{t}} &= \condP{\mVec{x}\idx{t}}{\mVec{z}\idx{t},\mVec{z}\idx{1:t-1},\mVec{u}\idx{1:t}} = \frac{ \condP{\mVec{z}\idx{t}}{\mVec{x}\idx{t},\mVec{z}\idx{1:t-1},\mVec{u}\idx{1:t}}\cdot \condP{\mVec{x}\idx{t}}{\mVec{z}\idx{1:t-1},\mVec{u}\idx{1:t}}}{\condP{\mVec{z}\idx{t}}{\mVec{z}\idx{1:t-1}, \mVec{u}\idx{1:t}}}
\\
&= \mu \cdot \condP{\mVec{z}\idx{t}}{\mVec{x}\idx{t},\mVec{z}\idx{1:t-1},\mVec{u}\idx{1:t}}\cdot \condP{\mVec{x}\idx{t}}{\mVec{z}\idx{1:t-1},\mVec{u}\idx{1:t}}\,.
\end{split}
\end{equation}
Der Einfluss der bedingte Wahrscheinlichkeit $\condP{\mVec{z}\idx{t}}{\mVec{z}\idx{1:t-1}, \mVec{u}\idx{1:t}}$ wird dabei durch den Faktor $\mu$ ersetzt, der für die Normalisierung reserviert wird, um sicherzustellen, dass es sich bei $\pbel{\mVec{x}\idx{t}}$ um eine legitime Wahrscheinlichkeitsverteilung handelt. Eine weitere Vereinfachung basiert auf der Annahme, dass $\mVec{x}\idx{t}$ einen so genannten vollständigen Zustand darstellt, das heißt $\mVec{x}\idx{t}$ beinhaltet bereits alle vergangen Information. Aus diesem Grund verändern zusätzliche Informationen aus vergangen Mess- und Stellgrößen eine mit $\mVec{x}\idx{t}$ bedingte Wahrscheinlichkeit nicht, woraus 
\begin{equation}
\condP{\mVec{z}\idx{t}}{\mVec{x}\idx{t},\mVec{z}\idx{1:t-1},\mVec{u}\idx{1:t}} = \condP{\mVec{z}\idx{t}}{\mVec{x}\idx{t}}
\end{equation}
resultiert. Wird dieses Ergebnis in die ursprüngliche Gleichung eingesetzt, ergibt sich
\begin{equation}
\label{eq_contbayes2}
\begin{split}
\bel{\mVec{x}\idx{t}} &= \mu \cdot \condP{\mVec{z}\idx{t}}{\mVec{x}\idx{t}}\cdot \condP{\mVec{x}\idx{t}}{\mVec{z}\idx{1:t-1},\mVec{u}\idx{1:t}} 
\\
&= \mu \cdot \condP{\mVec{z}\idx{t}}{\mVec{x}\idx{t}}\cdot \pbel{\mVec{x}\idx{t}}\,.
\end{split}
\end{equation}
Mithilfe von weiteren Annahmen kann auch die Wahrscheinlichkeit $\pbel{\mVec{x}\idx{t}}$ ermittelt werden, wofür diese zunächst erweitert wird.
\begin{equation}
\label{eq_pbel}
\begin{split}
\pbel{\mVec{x}\idx{t}} &= \condP{\mVec{x}\idx{t}}{\mVec{z}\idx{1:t},\mVec{u}\idx{1:t}}
\\
&= \int \condP{\mVec{x}\idx{t}}{\mVec{x}\idx{t-1},\mVec{z}\idx{1:t-1},\mVec{u}\idx{1:t}}\cdot \condP{\mVec{x}\idx{t-1}}{\mVec{z}\idx{1:t-1},\mVec{u}\idx{1:t}}d\mVec{x}\idx{t-1}\,.
\end{split}
\end{equation}
Falls $\mVec{x}\idx{t-1}$ ebenfalls ein vollständiger Zustand ist kann eine Vereinfachung der Form
\begin{equation}
\label{eq_sub1}
\condP{\mVec{x}\idx{t}}{\mVec{x}\idx{t-1},\mVec{z}\idx{1:t-1},\mVec{u}\idx{1:t}}=\condP{\mVec{x}\idx{t}}{\mVec{x}\idx{t-1},\mVec{u}\idx{t}}
\end{equation}
durchgeführt werden. Wenn zusätzlich angenommen werden kann, dass $\mVec{u}\idx{t}$ zufällig gewählt wird und somit kein Zusammenhang zwischen $\mVec{x}\idx{t-1}$ und $\mVec{u}\idx{t}$ besteht, folgt
\begin{equation}
\label{eq_sub2}
\condP{\mVec{x}\idx{t-1}}{\mVec{z}\idx{1:t-1},\mVec{u}\idx{1:t}} = \condP{\mVec{x}\idx{t-1}}{\mVec{z}\idx{1:t-1},\mVec{u}\idx{1:t-1}} = \bel{\mVec{x}\idx{t-1}}\,.
\end{equation}
Die Substitution der beiden Ergebnisse (\ref{eq_sub1}) und (\ref{eq_sub2}) in Gleichung (\ref{eq_pbel}) liefert
\begin{equation}
\label{eq_contbayes1}
\pbel{\mVec{x}\idx{t}} = \int \condP{\mVec{x}\idx{t}}{\mVec{x}\idx{t-1},\mVec{u}\idx{t}}\cdot \bel{\mVec{x}\idx{t-1}}d\mVec{x}\idx{t-1}
\end{equation}
und bildet zusammen mit Gleichung (\ref{eq_contbayes2}) die folgende Berechnungsvorschrift des Bayes-Filter.
\begin{lstlisting}[mathescape=true, caption={Bayes-Filter},captionpos=b]
Algorithm: BayesFilter($\bel{\mVec{x}\idx{t-1}},\mVec{u}\idx{t},\mVec{z}\idx{t}$):
	for all $\mVec{x}\idx{t}$ do:
		$\pbel{\mVec{x}\idx{t}} = \int \condP{\mVec{x}\idx{t}}{\mVec{x}\idx{t-1},\mVec{u}\idx{t}}\cdot \bel{\mVec{x}\idx{t-1}}d\mVec{x}\idx{t-1}$
		$\bel{\mVec{x}\idx{t}} = \mu \cdot \condP{\mVec{z}\idx{t}}{\mVec{x}\idx{t}}\cdot \pbel{\mVec{x}\idx{t}}$
	endfor
	return $\bel{\mVec{x}\idx{t}}$
\end{lstlisting}

\newpage
\subsection*{Applikation, Interpretation und kritische Anmerkungen}
Die bisherige Untersuchung hat sich auf die Herleitung der Berechnungsvorschrift und deren mathematischen Gültigkeit beschränkt, weshalb an dieser Stelle die Ergebnisse auf die konkrete Problemstellung der Lokalisierung übertragen werden. Dafür werden zunächst die Argumente des Algorithmus betrachtet: Bei den Vektoren $\mVec{u}\idx{t}$ und $\mVec{z}\idx{t}$ handelt es sich um die aktuellen Stell- und Messwerte. Der Erstgenannte enthält für gewöhnlich die aktuelle Geschwindigkeit des Roboters. In $\mVec{z}\idx{t}$ werden die aktuellen Sensorinformationen zusammengefasst, wobei es sich in diesem Fall um Distanzmessungen handelt, die mit einem Laserscanner gewonnen wurden. Das dritte Argument enthält die Wahrscheinlichkeit $\bel{\mVec{x}\idx{t-1}}$ am vorherigen Abtastzeitpunkt, die für den Fall $\mVec{x}\idx{0}$ initialisiert werden muss. Bei der Wahl der Startverteilung wird auf a priori Kenntnisse und Heuristiken zurückgegriffen. Beispielsweise kann der Anwender anhand einer Karte Positionen, die von einem Hindernis belegt sind, vorab ausschließen. Liegen keinerlei a priori Kenntnisse über die Ausgangsposition des Roboters vor, wird die Wahrscheinlichkeit $\bel{\mVec{x}\idx{0}}$ als Gleichverteilung initialisiert.

Als letzter Bestandteil des Filters müssen die bedingten Wahrscheinlichkeiten $\condP{\mVec{x}\idx{t}}{\mVec{x}\idx{t-1},\mVec{u}\idx{t}}$ und $\condP{\mVec{z}\idx{t}}{\mVec{x}\idx{t}}$ diskutiert werden. Die Verteilung $\condP{\mVec{x}\idx{t}}{\mVec{x}\idx{t-1},\mVec{u}\idx{t}}$ gibt an, wie wahrscheinlich eine Position $\mVec{x}\idx{t}$ ausgehend von der Position $\mVec{x}\idx{t-1}$ mit dem Stellvektor $\mVec{u}\idx{t}$ erreicht werden kann. Es liegt also ein stochastisches Bewegungsmodell vor. Im Gegensatz zu seinem deterministischen Pendant wird anhand der Position $\mVec{x}\idx{t-1}$ und Geschwindigkeit $\mVec{x}\idx{t}$ keine exakte Position $\mVec{x}\idx{t}$ berechnet, sondern eine probabilistische Aussage darüber getroffen, mit welcher Wahrscheinlichkeit die gegebene Position $\mVec{x}\idx{t}$ erreicht werden wird. 

Eine ähnlich Rolle übernimmt die Wahrscheinlichkeit $\condP{\mVec{z}\idx{t}}{\mVec{x}\idx{t}}$, die ein Messmodell darstellt. Eine deterministische Analogie besteht in der Ausgangsgleichung eines Zustandsraummodells
\begin{equation}
\mVec{y} = C(\mVec{x})\,,
\end{equation}
bei der anhand des Zustandsvektors $\mVec{x}$ ein Ausgangsvektor $\mVec{y}$ berechnet wird, der sich in der Regelungstechnik für gewöhnlich aus den gemessenen Größen zusammensetzt. Ähnlich wie bei dem Bewegungsmodell soll bei einem stochastischem Messmodell keine exakte Berechnung des Messvektors $\mVec{z}\idx{t}$ durchgeführt werden - insbesondere weil dieser bereits vorliegt. Die Aufgabe besteht vielmehr darin, zu schätzen wie wahrscheinlich ein gemessener Vektor $\mVec{z}\idx{t}$ aus einem gegebenen Zustand $\mVec{x}\idx{t}$ hervorgeht. Beispielsweise lässt sich mit recht hoher Wahrscheinlichkeit schließen, dass der Roboter unmittelbar vor einem Hindernis steht, wenn alle Abstandsmessungen sehr kleine Werte liefern.

Aus der Sicht des Applikators stellen die Verteilungen $\condP{\mVec{x}\idx{t}}{\mVec{x}\idx{t-1},\mVec{u}\idx{t}}$ und $\condP{\mVec{z}\idx{t}}{\mVec{x}\idx{t}}$ nicht nur Modelle sondern die relevanten Stellschrauben dar, um die Ergebnisse und Qualität des Filteralgorithmus zu beeinflussen. Folglich besteht die Hauptaufgabe darin, das stochastische Bewegungs- und Messmodell an die gegebenen Rahmenbedingungen anzupassen. Hier stellt sich die Frage, welche Form die Qualität eines stochastischen Modells annimmt. Aus der deterministischen Modellbildung ging bereits die Erkenntnis hervor, dass sich qualitativ hochwertige Modelle nicht durch einen unendlichen Detaillierungsgrad und der damit einhergehenden Genauigkeit auszeichnen. Das Gegenteil ist der Fall: Die hohe Kunst der Modellbildung besteht darin einen Kompromiss zwischen Detail und Aussagekraft zu finden. Detailgetreue Modellierungstiefe bringt als unvermeidlichen Nachteil Komplexität mit sich; eine Komplexität, welche die Handhabung des Modells ungemein erschwert und die Nutzung des Modells behindert. Diese Aspekte motivieren die Philosophie einer Modellierungsform, die sich auf die markanten Charakteristika eines Prozesses beschränkt. Die für die Aufgabenstellung relevanten Eigenschaften geben die erforderliche Detailierungstiefe des Modells vor. 

In diesem Sinne stellt das stochastische Modell aus mehreren Gründen eine mächtigen Ansatz dar. Während bei deterministischen Modellen die Vereinfachungen und Vernachlässigungen lediglich in Form von Annahmen in der begleitenden Dokumentation artikuliert werden können, bietet ein probabilistischen Modell die Möglichkeit Ungenauigkeiten als Wahrscheinlichkeit auszudrücken. Ebenso können heuristische Argumente, die sich nur schwer in einem deterministischen Kontext formulieren lassen, mithilfe einer probabilistischen Abschätzung in das mathematische Modell aufgenommen werden. Dieses Argument kommt besonders dann zum Tragen, wenn eine Problemstellung betrachtet wird, für die keinerlei physikalischen Gesetze bestehen. Beispielsweise steht im Fall des Bewegungsmodells das elaborierte Feld der technischen Mechanik bereit, um exakte Bewegungsgleichungen der Roboterdynamik zu formulieren. Lediglich deren unhandliche Komplexität motiviert die Verwendung simpler stochastischer Modelle. Eine andere Situation ergibt sich im Falle des Messmodells: Angenommen es werden Stereokameras verwendet, um die Distanz der umgebenden Hindernisse zu ermitteln. Es ist praktisch unmöglich einen exakten Zusammenhang zwischen der Umgebung, dem Messprinzip der Stereokameras und dem resultierenden Messvektor herzustellen. Somit bleibt dem Anwender keine andere Wahl als das Modell anhand von Heuristiken zu konstruieren, die wiederum auf stark vereinfachten Modellen und Annahmen basieren. 

Aus diesen Betrachtungen folgt bereits ein wichtiger Gesichtspunkt für die Beurteilung der Filtermethoden: Ein schlechtes Ergebnis kann unter Umständen nicht von dem Algorithmus sondern von den verwendeten Modellen verschuldet sein. Insofern muss Vorsicht bei der Auswertung und Beurteilung von Experimenten walten, um ein vorzeitiges Verwerfen potenter Ansätze zu vermeiden. Aus diesem Blickwinkel sollen nun auch die bei der Herleitung des Bayes-Filter getroffenen Annahmen betrachtet werden. Es kann recht leicht argumentiert werden, weshalb die Annahmen nicht zutreffen können. 

Beispielsweise folgte aus der Vollständigkeit des Zustandes $\mVec{x}\idx{t}$ die Gleichung
\begin{equation}
\condP{\mVec{z}\idx{t}}{\mVec{x}\idx{t},\mVec{z}\idx{1:t-1},\mVec{u}\idx{1:t}} = \condP{\mVec{z}\idx{t}}{\mVec{x}\idx{t}}\,,
\end{equation}
welche wiederum bedeutet, dass die Messung $\mVec{z}\idx{t-1}$ nicht von vergangen Messungen abhängt. Diese Umstand trifft nicht mehr zu sobald das Messglied einer PT1-Dynamik unterliegt.
{\color{red} Das Vollständigkeitsargument kann auch so verstanden werden, dass zt schon von zt-1 abhängen darf, diese Abhängikeit aber bereits in xt steckt}

Die zweite Annahme bestand darin, dass die Stellgröße $\mVec{u}\idx{t}$ zufällig gewählt wird, und somit nicht von der Position $\mVec{x}\idx{t}$ abhängt. Diese Annahme ist vollkommen haltlos, da die Lokalisierung im Rahmen dieser Arbeit nur verfolgt wird, um das Ergebnis $\mVec{x}\idx{t}$ zur Berechnung einer Stellgröße $\mVec{u}\idx{t}$ zu verwenden. Zur Verteidigung des Verfahrens wird die Ungenauigkeit der Modelle aufgeführt, wobei argumentiert wird, dass die Modellfehler den Wiederstoß gegen die bestehenden Annahmen überwiegen. Allerdings ist kritisch zu vermerken, dass es sich bei diesem Argument auch lediglich um eine Annahme handelt. An dieser Stelle offenbart sich eine Schwachstelle der stochastischen Ansätze: Es ist kaum möglich eine analytische Beurteilung der Verfahren durchzuführen. Der Vorteil, dass heuristische Argumente in Form von Wahrscheinlichkeiten ausgedrückt werden, verhindert gleichermaßen ein deterministisches, also ein absolutes Urteil. Als entscheidendes Kriterium bleibt lediglich das Experiment, wobei es wieder schwer fällt zwischen den Einflüssen des Algorithmus und der probabilistischen Modelle zu differenzieren.

