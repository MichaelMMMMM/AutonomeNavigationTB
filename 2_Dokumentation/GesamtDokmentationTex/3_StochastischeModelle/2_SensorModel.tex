\section{Messmodell}\footnote{Diser Abschnitt orientiert sich inhaltlich an \cite[S. 153 ff]{ProbRob}}
Als weiteres Beispiel wird ein stochastisches Messmodell vorgestellt, das später bei der Lokalisierung wiederverwendet wird. Das Ziel des Messmodells ist es den Zusammenhang zwischen einem Zustandsvektor $\mVec{x}\idx{t}$ und einem Messvektor $\mVec{z}\idx{t}$ in Form der bedingten Wahrscheinlichkeit
\begin{equation}
\condP{\mVec{z}\idx{t}}{\mVec{x}\idx{t}}
\end{equation}
herzustellen. Die Verteilung sagt aus, mit welcher Wahrscheinlichkeit ein Messvektor $\mVec{z}\idx{t}$ von einer vorgegebenen Position $\mVec{x}\idx{t}$ aus erzielt wird. Das Messmodell dient als Paradebeispiel dafür, wie heuristische Argumente in stochastische Modelle einfließen können. Unabhängig von dem exakten Sensor wird angenommen, dass es sich um ein auf Messstrahlen basiertes Messprinzip handelt. Das heißt der Messvektor $\mVec{z}\idx{t}$ setzt sich aus mehreren Messwerten $z^k\idx{t}$ zusammen, die jeweils die auf einem Strahl gemessene Distanz wiedergeben. 

Als erste Störgröße wird ein gewöhnliches Messrauschen eingeführt, das als normal verteilt angenommen wird. Der Mittelwert der Verteilung entspricht der tatsächliche Distanz auf dem Messstrahl, die mit $z^{k*}\idx{t}$ denotiert wird. Als zweiter Parameter muss die Varianz $\sigma\idx{hit}$ festgelegt werden, die angibt wie stark der Messwert von dem Rauschen beeinflusst wird. Somit ergibt sich als Modell für das Sensorrauschen
\begin{equation}
p\idx{hit}\left( z^{k}\idx{t} \mCond \mVec{x}\idx{t} \right) = \left\{ \begin{array}{cl}
\eta \cdot \mathcal{N}(z^k\idx{t}, z^{k*}\idx{t}, \sigma^2\idx{hit}) & \hspace{5mm}\forall z^k\idx{t} \in \left[0, z\idx{max}\right] \\
0 & \hspace{5mm} \forall z^k\idx{t} \notin \left[0, z\idx{max}\right]
\end{array}\right. \,.
\end{equation}
Die Messwerte werden außerdem von unerwarteten Objekten beeinflusst. Angenommen es liegt eine Karte der Umgebung vor und es bewegt sich ein Mensch innerhalb des Raumes. Wenn der Mensch nun die Messstrahlen unterbricht, tritt ein geringerer Messwert als anhand der Karte erwartet ein. Im Allgemeinen gilt, dass im Falle von Störungen durch unerwartete Objekte die Messwerte lediglich kleiner, aber niemals größer werden. Des Weiteren nimmt die Wahrscheinlichkeit, dass ein Messwert von einem Objekt beeinflusst wird mit zunehmendem Messdistanz ab. Für das Modell wird eine exponentiell abfallende Verteilung
\begin{equation}
\renewcommand*{\arraystretch}{1.3}
p\idx{short}\left( z^k\idx{t} \mCond \mVec{x}\idx{t} \right) = \left\{ \begin{array}{cl}
\eta \cdot \lambda\idx{short}\cdot e^{-\lambda\idx{short}\cdot z^{k}\idx{t}} & \hspace{5mm} \forall z^{k}\idx{t} \in \left[0, z^{k*}\idx{t}\right] \\
0 & \hspace{5mm} \forall z^{k}\idx{t} \notin \left[0, z^{k*}\idx{t}\right]
\end{array}\right. 
\end{equation}
gewählt, wobei $\lambda\idx{short}$ als wählbarer Parameter die Häufigkeit der Störungen widerspiegelt. Die Funktion $p\idx{short}$ besagt, wie mit welcher Wahrscheinlichkeit es sich bei dem Messwert $z^k_t$ um einen, auf Grund eines unerwarteten Hindernisses, zu kurze Messung handelt.

Als dritte Ursache für Messfehler wird das sporadische Versagen des Sensors herangezogen. Derartige Phänomene können bei Laserscannern auftreten, wenn beispielsweise schwarze Oberflächen die Strahlen vollständig absorbieren. Ähnliche Probleme werden bei Sonarsensoren durch Interferenz oder schallabsorbierende Medien hervorgerufen. Unabhängig von dem Messprinzip des Sensors werden in all den beschriebenen Fällen - fälschlicherweise - maximale Distanzen gemessen, wodurch die Verteilung
\begin{equation}
p\idx{max}\left(z^k\idx{t} \mCond \mVec{x}\idx{t}\right) = \left\{ \begin{array}{cl}
1 & \hspace{5mm} \forall z = z\idx{max} \\
0 & \hspace{5mm} \forall z  \neq z\idx{max}
\end{array}\right.
\end{equation}
motiviert wird. Die Aussage der Zufallsvariable besteht darin, dass im Falle eines maximalen Messwertes von einem Messfehler ausgegangen werden kann

Als letzter  Fall werden rein zufällig falsche Phänomene betrachtet. Als Argument dient die Beobachtung, dass es unerklärliche Messwerte gibt, die - zu Gunsten der Simplizität - durch eine Gleichverteilung der Form
\begin{equation}
\renewcommand*{\arraystretch}{1.3}
p\idx{rand}\left( z^k\idx{t} \mCond \mVec{x}\idx{t}\right) = \left\{ \begin{array}{cl}
\frac{1}{z\idx{max}} & \hspace{5mm} \forall z^k\idx{t} \in \left[0, z\idx{max}\right] 
\\
0 & \hspace{5mm} \forall z^k\idx{t} \notin \left[0, z\idx{max}\right]
\end{array} \right.
\end{equation}
repräsentiert werden. 

Die gesuchte Wahrscheinlichkeit $\condP{z^k\idx{t}}{\mVec{x}\idx{t}}$ wird aus Summe der vier Verteilung berechnet, wobei diese mit den Faktoren $z\idx{hit}$, $z\idx{short}$, $z\idx{max}$ und $z\idx{rand}$ gewichtet werden. Die Gewichtung erlauben es dem Anwender, die Heuristiken nach ihrer Glaubhaftigkeit zu bewerten.
\begin{equation}
\condP{z^k\idx{t}}{\mVec{x}\idx{t}} = \begin{bmatrix} z\idx{hit} & z\idx{short} & z\idx{max} z\idx{rand} \end{bmatrix} \cdot \begin{bmatrix}
p\idx{hit}\left( z^{k}\idx{t} \mCond \mVec{x}\idx{t} \right) \\
p\idx{short}\left( z^k\idx{t} \mCond \mVec{x}\idx{t} \right) \\
p\idx{max}\left(z^k\idx{t} \mCond \mVec{x}\idx{t}\right) \\
p\idx{rand}\left( z^k\idx{t} \mCond \mVec{x}\idx{t}\right)
\end{bmatrix}\,.
\end{equation}