\chapter{Aufbau des Gesamtsystem und Infrastruktur}
\pdfcomment{Wofür?}In der Endvorstellung des Projektes navigieren vier Roboter simultan durch denselben Raum, woran recht leicht ersichtlich wird, dass ein Weg zum Datenaustausch zwischen den Robotern geschaffen werden muss. Auch eine zentrale Steuereinheit soll in der Lage sein, alle Roboter anzusprechen und zu dirigieren. Selbst ein einzelner Roboter stellt bereits einige Anforderungen an die Kommunikationsstruktur: Da eine erfolgreiche Navigation auf dem Zusammenspiel mehrerer komplexer Algorithmen basiert, müssen externe Analysetools relevante Daten abgreifen, die zwischen den Komponenten ausgetauscht werden. Diese Form des Debuggings ist unerlässlich, um das Verhalten des Roboters nachvollziehen zu können. In diesem Projekt erfüllt ROS diese Anforderungen, weshalb in den folgenden Abschnitten das Kommunikationskonzept unter ROS erläutert wird. Auch die Simulations- und Analysewerkzeuge, die in der Arbeit zum Einsatz kommen, werden vorgestellt und deren ROS-Schnittstellen erläutert.

\pdfcomment{ROS-Netzwerk}Jeder TurtleBot ist mit einem Mini-PC ausgestattet, auf denen Ubuntu 14.04 und ROS-Indigo installiert sind. Als zentrale Steuereinheit des Roboterverbundes fungiert ein weiterer Mini-PC, der über WLA-Netzwerk mit den Roboter-PCs verbunden ist. Das private Netzwerk wird mittels eines TP-Link Routers verwaltet, in dem auch weitere Entwicklungsrechner beitreten können. Auf dem Master-PC wird \lstinline{ros_core}{} ausgeführt, über den die Kommunikation abläuft. Als Kommunikationsmittel werden in ROS Nachrichten verwendet, die jeweils unter einer so genannten \lstinline{topic}{} veröffentlicht werden. Beispielsweise versendet der Laserscanner zyklisch eine Nachricht des Typs \lstinline{sensor_msg/laser_scan}{} unter der topic \lstinline{/scan}{}. Alle Nodes, die an den Sensordaten interessiert sind, abonnieren die Topic. Indem Analysewerkzeuge wie MATLAB relevante Topics mithören, können die zugehörigen Daten aufgezeichnet und visualisiert werden. Umgekehrt ist es auch möglich den TurtleBot vollständig durch eine Simulation zu ersetzen. Gazebo veröffentlicht alle Topics, die für gewöhnlich von dem TurtleBot versendet werden, wodurch dieser ersetzt wird. Die restlichen Bestandteile des Netzwerkes bleiben erhalten, weshalb der Wechsel zwischen Realität und Simulation mühelos abläuft.

\pdfcomment{Simulationswerkzeug: Gazebo} Die Idee Gazebo in Verbindung mit ROS für die Simulation zu verwenden entstammt der Arbiet \cite{ROSGazebo}. Dort wurde unter einer ähnlichen Konfiguration ein P3-DX Roboter verwendet. Der primäre Vorteil dieser Simulationsstruktur besteht darin, dass die Algorithmen und Konfigurationen unverändert von der Simulation auf den Roboter übertragen werden können. Gazebo bietet auch für die TurtelBots eine vollständige Unterstützung, das heißt es besteht eine frei zugängliche Integration des TurtleBot in Gazebo. Allerdings existiert keine Implementation des hier verwendeten Laserscanner, Tim551, weshalb die Sensorik nicht exakt abgebildet werden kann. Prinzipiell können die Sensoren eingepflegt werden, was jedoch im Rahmen dieser Arbeit aus zeitlichen Gründen nicht erfolgt ist. Daraus resultiert der Nachteil, dass die Simulation nur beschränkt Schlüsse auf die Realität zulässt. Nichtsdestotrotz ergeben sich zwei wichtige Vorteile: Einerseits können die verschiedenen Konfigurationen des ROS-Navigation-Stack anhand der Simulation erprobt und auf ihre Richtigkeit überprüft werden, bevor sie auf reale Anwendungsfälle übertragen werden. Andererseits können im Rahmen der Simulation die Komponenten des Navigation-Stack vollkommen entkoppelt werden. Beispielsweise hängen die Ergebnisse der Navigations-Algorithmen stark von der Qualität der Lokalisierung ab, weshalb die Planungs- und Lokalisierungsproblematik in einem realen Umfeld nicht getrennt untersucht werden können. In der Simulation können Positionsdaten jedoch unmittelbar abgegriffen und der Navigation zur Verfügung gestellt werden. Insofern stellt Gazebo ein mächtiges Werkzeug für die schrittweise Inbetriebnahme des Navigation-Stacks dar.

\pdfcomment{Simulationswerkzeug: MATLAB}MATLAB kann im Rahmen der Robotics-Toolbox ebenfalls als ROS-Node betrieben werden, wodurch eine Schnittstelle zwischen MATLAB und ROS geschaffen wird. Hier bringt MATLAB den Vorteil mit sich, dass es für die Implementierung von Algorithmen oftmals besser geeignet ist als C++ oder auch Python. Insbesondere die Möglichkeiten im Bereich der Visualisierung von Karten und Plots erweisen sich als mächtige Werkzeuge bei der Implementierung und Erprobung von Algorithmen. Außerdem können simple Anwendungsszenarien, wie sie zum Beispiel im Rahmen der diskreten Planung auftreten, vollständig mit MATLAB simuliert werden.

\pdfcomment{Visualisierungswerkzeug: RViz}Als letztes Werkzeug sei an dieser Stelle RViz genannt, das als ROS-Paket vorliegt und für die Visualisierung von Robotikanwendungen konzipiert wurde. Mithilfe von RViz können sowohl Karten als auch Roboter und Pfade graphisch dargestellt werden. Da RViz für die Anwendung mit ROS konzipiert wurde, stehen vollständige Konfiguration für den Einsatz mit der autonomen Navigation zur Verfügung, die Out-of-the-Box genutzt werden können.