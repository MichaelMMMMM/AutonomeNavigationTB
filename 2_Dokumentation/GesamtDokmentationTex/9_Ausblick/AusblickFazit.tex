\chapter{Ausblick und Fazit}
Im Rahmen der vorliegenden Arbeit wurde das Funktionsprinzip des ROS-Navigation-Stack erläutert und anhand von drei Experimenten bewiesen. Im ersten wurden die Grundfunktionen wie die Planung und Verfolgung des globalen Pfades nachgewiesen. Das zweite Experimente diente zur Illustration der Hindernisdetektion und der Planung bzw. Ausführung eines Ausweichmanövers. Zuletzt wurde das Konzept der Lokalisierung anhand eines dritten Experiments bewiesen, indem nur eine ungenaue Positionsschätzung bekannt war. Hier hat der Roboter innerhalb kurzer Zeit eine sichere und präzise Schätzung seiner Position erarbeitet, womit die Fähigkeit der Lokalisierung bewiesen wurde. Insofern wurden sämtliche Anforderungen an die Navigation erfüllt.


Allerdings wurde in diesen Experimenten nur die Navigation eines einzelnen Roboters untersucht. Das Ergebnis nicht auf die parallele Navigation mehrerer Roboter übertragen werden. Der Grund hierfür liegt in den Namenskonflikten der ROS-Transformationen zwischen den Robotern, die nicht vollständig aufgelöst werden konnten. In dieser Arbeit wurde versucht, mittels des ROS-Parameter \lstinline{tf_prefix}{} den Transformationen der verschiedenen Robotern Namensräume zuzuweisen, wodurch eine klare Trennung erfolgt. Allerdings hat dieser Schritt nicht die erwarteten Ergebnisse geliefert. Die exakten Ursachen des Fehlverhalten konnten in dieser Arbeit nicht geklärt werden.

Hieraus resultiert auch das offensichtliche Ziel für eine Folgearbeit: Es muss ein Konzept entwickelt werden, um die Namenskonflikte unter ROS aufzulösen.

Da die Roboter sich gegenseitig als Hindernisse wahrnehmen und somit in ihrer Pfadplanung beachten, ist zu erwarten, dass das hier vorgestellte Konzept auch bei der simultanen Navigation mehrerer Roboter funktioniert. Eine gekoppelte Navigation der Roboter sollte nicht erforderlich sein.