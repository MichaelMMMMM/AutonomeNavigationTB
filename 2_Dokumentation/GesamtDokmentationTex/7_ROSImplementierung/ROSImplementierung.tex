\chapter{ROS Implementierung}
Nachdem in den vorherigen Kapiteln die theoretischen Grundlagen der Navigationsalgorithmen erläutert wurden, widmet sich dieser Teil der Arbeit der Umsetzung der Konzepte in ROS. Zunächst wird der Aufbau des Systems, Anordnung der Rechner und die Netzwerkkonfiguration betrachtet. Im Anschluss werden die Roboter schrittweise in Betrieb genommen, wobei die relevanten Dateien erklärt werden und  ein Schritt für Schritt Anleitung gezeigt wird.

Als erste Demonstration wird die Fernsteuerung der Roboter betrachtet, wofür eine Launch-Datei zur Initialisierung der Turtlebots angelegt wird. Außerdem wird auf dem Master eine Anwendung ausgeführt, um den Roboter per Tastatur zu steuern.
Im nächsten Schritt wird die Kartenaufzeichnung implementiert. Hier wird auf dem Master-PC zusätzlich das Paket \lstinline{hector_mapping}{} ausgeführt, welches die Karte erstellt. Außerdem wird \lstinline{Rviz}{} verwendet, um die aktuelle Karte und Roboterposition zu visualisieren.
Im dritten Anwendungsbeispiel wird die zuvor aufgezeichnete Karte für die autonome Navigation genutzt. Hierfür wird auf Seite des Master die vollständigen Navigationsalgorithmen konfiguriert und ausgeführt. Außerdem zeigt \lstinline{Rviz}{} in diesem Beispiel nicht nur die Karte und den Roboter an, sondern dient auch als Eingabemöglichkeit für die Zielposition.

Im Anschluss wird der Endszenario diskutiert, in dem zwei Robotern simultan navigieren. Hier entsteht auf ROS-Seite das Problem, dass Konflikte in der Namensgebung von Nachrichten und Topics entstehen. Eine mögliche Lösung stellen die unter ROS verfügbaren Namensräume dar, wodurch die Daten der beiden Roboter entkoppelt werden können. Allerdings sind in der Umsetzung dieses Lösungsansatzes Probleme mit der Lokalisierung aufgetreten, die bisher noch nicht erklärt bzw. gelöst werden konnten. An dieser Stelle wird der Fehlerfall aufgezeigt und mögliche Lösungen analysiert.

\newpage
\section{Systemstruktur}
\pdfcomment{Formulierung, die Erklärung von Master/Slave ist schlecht}
In dieser Arbeit wird die Navigation von maximal zwei Robotern zur selben Zeit betrachtet. Jeder der Roboter ist mit einem PC ausgestattet, die über WLAN mit einem TP-Link-Router verbunden sind, der als Access-Point des Netzwerks fungiert. Der Roboterverbund ist nach einem Master-Slave-Prinzip konzipiert, wobei die Roboter bzw. deren PCs als Slave agieren. Ein weiterer PC ist mit dem Router verbunden und agiert als Master. Der Master-PC ist zusätzlich mit Tastatur, Maus und Bildschirm ausgestattet und stellt somit die Steuereinheit des Systems dar. Daher stammt auch die Bezeichnung Master-PC, da von diesem die Befehlskette bedient wird. Die Slave-PCs, welche sich auf den Robotern befinden, nehmen die Befehle entgegen, arbeiten also als Slaves \pdfcomment{really?}.

Das Netzwerk läuft in dem IP-Adressbereich 192.168.0.X, wobei die Adressen des Master- und der Slave-PCs statisch zugewiesen werden. Der Master-PC ist unter der Adresse 192.168.0.100 erreichbar; Roboter R2 unter der Adresse 192.168.0.102; Roboter R4 unter der Adresse 192.168.0.104. Außerdem ist es möglich, dass sich weitere Rechner mit dem Netzwerk verbinden, wodurch auf sämtliche Daten des ROS-Netzes zugegriffen werden kann. Somit können auch Entwicklungswerkzeuge wie MATLAB verwendet werden, um die Daten abzugreifen und weiterzuverarbeiten.

Das Netzwerk trägt den Namen \lstinline{EML_Turtlebot_NET}{} mit dem Passwort \lstinline{turtlebot}{}. Die statische Konfiguration der IP-Adressen kann im Detail in \cite[S. 23]{Turtleboys} nachgelesen werden.
Auf jedem der Rechner existiert ein Nutzer mit dem Namen \lstinline{turtlebot}{} und Passwort \lstinline{turtlebot}{}, der genutzt werden kann, um eine ssh-Verbindung aufzubauen.

\newpage
\section{Inbetriebnahme der Turtlebots}
In diesem Teil der Arbeit wird die praktische Umsetzung der Navigationsalgorithmen diskutiert, wofür ROS verwendet wird. Unter ROS werden so genannte Launch-Dateien genutzt, um die beteiligten Pakete zu parametrisieren und zu starten. Typischerweise werden in jedem ROS-Projekt mehrere Launch-Dateien angelegt, die für die jeweiligen Anwendungsfälle ausgelegt sind. In dieser Arbeit werden zwei verschiedene Projekte angelegt. Auf dem Master-PC wird das Projekt \lstinline{EML_Navigation_Master}{} genutzt; auf den Slave-PCs das Projekt \lstinline{EML_Navigation_Slave}{}. Die Projektordner sind jeweils auf den Desktops der PCs zu finden. In dem Master-Projekt werden die vier Launch-Dateien
\begin{itemize}
\item \lstinline{EML_Mapping_Master.launch}{}
\item \lstinline{EML_Navigation_Master.launch}{}
\item \lstinline{EML_Navigation_Robot2_Master.launch}{}
\item \lstinline{EML_Navigation_Robot4_Master.launch}{}
\end{itemize}
genutzt, wobei die erste für die Kartenerstellung, die zweite für die Navigation eines einzelnen Roboters und die beiden letzten für die simultane Navigation der beiden Roboter verwendet werden.

In den Slave-Projekten sind die fünf Launch-Dateien
\begin{itemize}
\item \lstinline{EML_Hardware_Init_Slave.launch}{}
\item \lstinline{EML_Mapping_Slave.launch}{}
\item \lstinline{EML_Navigation_Slave.launch}{}
\item \lstinline{EML_Navigation_Robot2_Slave.launch}{}
\item \lstinline{EML_Navigation_Robot4_Slave.launch}{}
\end{itemize}
zu finden. Die Erste führt die Initialiserung der vorhandenen Hardware durch. Die vier weiteren Dateien stellen das Pendant zu den oben beschriebenen Master-Dateien dar.

\newpage
\subsection{Fernsteuerung eines Turtlebots}
Im aller ersten Schritt werden die beiden Roboter in Betrieb genommen, wofür die Fernsteuerung der beiden Geräte als Demonstration dienen soll. Auf den Slaves wird dafür die Launch-Datei \lstinline{EML_Hardware_Init_Slave.launch}{} angelegt, die die vorhandene Hardware initialisiert, sodass der Roboter über die entsprechenden ROS-Nachrichten bewegt werden kann.
\begin{lstlisting}[caption={EML\_Hardware\_Init\_Slave.launch},captionpos=b]
<launch>
	<!-- Basis-Initialisierung -->
	<include file="$(find turtlebot_bringup)/ ...
			launch/minimal.launch"/>
	
	<!-- Initialisierung des SICK-TIM551 -->
	<param name="robot_description" command="$(find xacro)/ ...
			xacro.py '$(find sick_tim)/urdf/example.urdf.xacro'"/>
	<include file="$(find turtlebot_chor_navigation)/launch/ ...
			include/sick_tim551_2050001_timefix.launch"/>
</launch>
\end{lstlisting}
In der Launch-Datei wrid zunächst eine Basisinitialisierung durchgeführt, wofür die Datei \lstinline{minimal.launch}{} aus dem offiziellem ROS-Packet \lstinline{turtlebot_bringup}{}\cite{WikiTurtlebotBringup} inkludiert wird. Als Zweites wird mithilfe einer der Launch-Datei \lstinline{sick_tim_551_2050001_} \lstinline{timefix.launch}{} aus dem Vorgängerprojekt \cite{Turtleboys} der SICK-Laserscanner initialisiert.

Um den Turtelbot über den Master-PC zu steuern, wird die Launch-Datei auf dem Slave ausgeführt.
\begin{lstlisting}[language=bash]
roslaunch EML_Navigation_Slave EML_Hardware_Init_Slave.launch
\end{lstlisting}
Die Eingabe der Steuerbefehle erfolgt auf dem Master-PC, wofür auf das Paket \lstinline{turtle}{} \lstinline{bot_teleop}{} \cite{WikiTurtlebotTeleop} zurückgegriffen wird. Zum Start wird der Befehl
\begin{lstlisting}
roslaunch turtlebot_teleop keyboard_teleop.launch
\end{lstlisting}
auf dem Master-PC ausgeführt. Daraufhin kann der Roboter mittels der Tastatur bewegt werden.

\newpage
\subsection{Aufzeichnung einer Karte}
Im nächsten Schritt wird die Kartographierung der Umgebung betrachtet, wobei die Ergebnisse der Vorgängerarbeit \cite[S. 47, ff]{Turtleboys} verwendet werden können. Auf Slave-Seite wird die Launch-Datei \lstinline{EML_Mapping_Slave.launch}{} verwendet, in zunächst die bereits angesprochene Datei zur Hardware-Initialisierung inkludiert wird. Im Anschluss werden die Launch-Dateien für die Kartenaufzeichnung des Pakets \lstinline{hector_mapping}{} \cite{WikiHector} eingebunden.
\begin{lstlisting}[caption={EML\_Mapping\_Slave.launch},captionpos=b]
<launch>
	<!-- Hardware-Initialisierung -->
	<include file="$(find EML_Navigation_Slave)/launch/ ... 
			EML_Hardware_Init_Slave.launch" />

	<!-- Hector-Mapping -->
	<node name="hector_mapping" pkg="hector_mapping" ...
			type="hector_mapping" output="screen">
        	<param name="base_frame" value="base_link"/>
        	<param name="odom_frame" value="base_link"/>
	        <param name="map_resolution" value="0.05"/>
	        <param name="map_size" value="2048"/>
	        <param name="map_start_x" value="0.5"/>
        	<param name="map_start_y" value="0.5"/>
	        <param name="map_update_distance_thresh" value="0.4"/>
	        <param name="map_update_angle_thresh" value="0.02"/>
	        <param name="map_pub_period" value="0.05"/>
	        <param name="map_multi_res_levels" value="2"/>
	        <param name="update_factor_free" value="0.4"/>
	        <param name="update_factor_occupied" value="0.9"/>
	        <param name="laser_min_dist" value="0.4"/>
	        <param name="laser_max_dist" value="10"/>
	        <param name="laser_z_min_value" value="-1.0"/>
	        <param name="laser_z_max_value" value="1.0"/>
	        <param name="pub_map_odom_transform" value="true"/>
	        <param name="output_timing" value="false"/>
	        <param name="scan_subscriber_queue_size" value="1"/>
	        <param name="pub_map_scanmatch_transform" ...
	        		value="true"/>
	        <param name="pub_map_scanmatch_transform" ...
	        		value="true"/>
	        <param name="tf_map_scanmatch_transform_ ...
	        		frame_name" value="scanmatcher_frame"/>
	    </node>
    <include file="$(find hector_geotiff)/launch/ ...
    		geotiff_mapper.launch"> </include>
    <node pkg="tf" type="static_transform_publisher" ...
    		name="map_nav_broadcaster" args="0.12 0 0 0 0 0 ...
    		base_link laser 100"/>
</launch>
\end{lstlisting}
Auf dem Master wird die Launch-Datei \lstinline{EML_Mapping_Master.launch}{} verwendet, die inhaltlich ebenfalls aus dem Vorgängerprojekt \cite{Turtleboys} übernommen ist. In der Datei wird lediglich \lstinline{Rviz}{} gestartet, um die Ergebnisse der Kartenerstellung zu visualisieren.
\begin{lstlisting}[caption={EML\_Mapping\_Master.launch},captionpos=b]
<launch>
	<!-- Starte Rviz, um Karte anzuzeigen. -->
	<param name="robot_description" command="$(find xacro)/ ...
			xacro.py '$(find sick_tim)/urdf/example.urdf.xacro'" />
	<node pkg="rviz" type="rviz" name="rviz" args="-d $(find ...
			turtlebot_chor_navigation)/rviz_cfg/rviz_sick.rviz"/>
</launch>
\end{lstlisting}
Neben den beiden Launch-Dateien muss wieder eine Anwendung zur Steuerung des Turtlebots gestartet werden. Außerdem muss am Ende der Kartenaufzeichnung die Applikation \lstinline{map_server}{} ausgeführt werden, um die Karte abzuspeichern. Es ergibt sich die folgende Anleitung für die Kartenaufzeichnung. Die einzelnen Schritten sind in \cite[S. 47 ff]{Turtleboys} ausführlich beschrieben.
\begin{lstlisting}[caption={Anleitung zur Kartenaufzeichnung},captionpos=b]
[M] roscore
[S] roslaunch EML_Navigation_Slave  EML_Mapping_Slave.launch
[M] roslaunch EML_Navigation_Master EMl_Mapping_Master.launch
[M] roslaunch turtelbot_teleop keyboard_teleop.launch
[M] rosrun map_server map_saver
\end{lstlisting}

\newpage
\subsection{Navigation eines einzelnen Roboters}
Nachdem eine Karte der Umgebung erstellt wurde, kann diese für die Navigation eines Roboters herangezogen werden. Auf dem Slave wurde dafür die Launch-Datei \lstinline{EML_Navigation_Slave.launch}{} angelegt, die lediglich eine Hardware-Initialisierung durchführt.
\begin{lstlisting}[caption={EML\_Navigation\_Slave.launch},captionpos=b]
<launch>
	<include file="$(find EML_Navigation_Slave)/launch/ ...
			EML_Hardware_Init_Slave.launch" />
</launch>
\end{lstlisting}
Auf dem Master wird die Launch-Datei \lstinline{EML_Navigation_Master.launch}{} verwendet, in der sämtliche Algorithmen der Navigation parametrisiert und gestartet werden.
\begin{lstlisting}[caption={EML\_Navigation\_Master.launch},captionpos=b]
<launch>
	<!-- Map-Server -->
	<arg name="map_file" default="$(find EML_Navigation_Master)...
			/maps/test_map.yaml"/>
	<node name="map_server" pkg="map_server" type="map_server" ...
			args="$(arg map_file)"/>
	<!-- AMC-Localization -->
	<arg name="3d_sensor" default="$(env TURTLEBOT_3D_SENSOR)"/>
	<arg name="initial_pose_x" default="0.0"/>
	<arg name="initial_pose_y" default="0.0"/>
	<arg name="initial_pose_a" default="0.0"/>
	<arg name="custom_amcl_launch_file" default="$(find ...
			turtlebot_navigation)/launch/includes/amcl/$...
			(arg 3d_sensor)_amcl.launch.xml"/>
	<include file="$(arg custom_amcl_launch_file)">
		<arg name="initial_pose_x" ...
				value="$(arg initial_pose_x)"/>
		<arg name="initial_pose_y" ...
				value="$(arg initial_pose_y)"/>
		<arg name="initial_pose_a" ...
				value="$(arg initial_pose_a)"/>
		<arg name="odom_frame_id" value="odom"/>
		<arg name="base_frame_id" value="base_footprint"/>
		<arg name="global_frame_id" value="map"/>
		<arg name="use_map_topic" value="true"/>
	</include>
	<!-- Static Transform from base_footprint to base_link -->
	<node pkg="tf" type="static_transform_publisher" ...
			name="base_link_footprint_tf_broadcaster"...
			args="0 0 .5 0 0 0 1 base_footprint base_link 100"/>
	<!-- Rviz -->
	<node name="rviz" pkg="rviz" type="rviz" args="-d ...
			$(find turtlebot_rviz_launchers)/rviz/navigation.rviz"/>
	<!-- move_base -->
	<include file="$(find EML_Navigation_Master) ...
			/launch/include/eml_move_base.launch"/>
</launch>
\end{lstlisting}

In der Datei wird zunächst die AMC-Lokalisierung gestartet, wofür die Standardparameter des ROS-Pakets \lstinline{amcl} \cite{WikiAMCL} genutzt werden. Des Weiteren wird \lstinline{Rviz}{} gestartet, womit die Karte, Position des Roboters sowie die Zielposition graphisch dargestellt werden. Als Konfiguration dient die Datei \lstinline{navigation.rviz}{}, die in dem ROS-Paket \lstinline{turtlebot_rviz_launchers} \cite{WikiRVIZLaunchers} enthalten ist. Zuletzt wird das Paket \lstinline{move_base} \cite{WikiMoveBase} gestartet, in dem die letztendliche Navigation berechnet wird. Als Konfigurationsdatei dient \lstinline{eml_move_base.launch}{}, welche die \lstinline{move_base}{}-Instanz parametrisiert.
\begin{lstlisting}[caption={eml\_move\_base.launch},captionpos=b]
<launch>
	<include file="$(find turtlebot_navigation)/launch/ ...
			includes/velocity_smoother.launch.xml"/>
	<include file="$(find turtlebot_navigation)/launch/ ...
			includes/safety_controller.launch.xml"/>
  
	<arg name="odom_frame_id"   default="odom"/>
	<arg name="base_frame_id"   default="base_footprint"/>
	<arg name="global_frame_id" default="map"/>
	<arg name="odom_topic" default="odom" />
	<arg name="laser_topic" default="scan" />
	<arg name="custom_param_file" default="$(find ...
			turtlebot_navigation)/param/dummy.yaml"/>

	<node pkg="move_base" type="move_base" respawn="false" ...
			name="move_base" output="screen">
	<rosparam file="$(find EML_Navigation_Master)/param/ ...
			eml_costmap_common_params.yaml" command="load" ...
			ns="global_costmap" />
	<rosparam file="$(find EML_Navigation_Master)/ ...
			param/eml_costmap_common_params.yaml"    ...
			command="load" ns="local_costmap" />   
	<rosparam file="$(find turtlebot_navigation)/param/ ...
			local_costmap_params.yaml" command="load" />   
	<rosparam file="$(find turtlebot_navigation)/param/ ...
			global_costmap_params.yaml" command="load" />
	<rosparam file="$(find EML_Navigation_Master)/param/...
			eml_dwa_local_planner_params.yaml" command="load" />
	<rosparam file="$(find turtlebot_navigation)/param/...
			move_base_params.yaml" command="load" />
	<rosparam file="$(find turtlebot_navigation)/param/...
			global_planner_params.yaml" command="load" />
	<rosparam file="$(find turtlebot_navigation)/param/...
			navfn_global_planner_params.yaml" command="load" />
	    
	<!-- reset frame_id parameters using user input data -->
	<param name="global_costmap/global_frame" ...
			value="$(arg global_frame_id)"/>
	<param name="global_costmap/robot_base_frame" ...
			value="$(arg base_frame_id)"/>
    <param name="local_costmap/global_frame" ...
    		value="$(arg odom_frame_id)"/>
    <param name="local_costmap/robot_base_frame" ...
    		value="$(arg base_frame_id)"/>
    <param name="DWAPlannerROS/global_frame_id" ...
    		value="$(arg odom_frame_id)"/>

    <remap from="cmd_vel" ...
    		to="navigation_velocity_smoother/raw_cmd_vel"/>
    <remap from="odom" to="$(arg odom_topic)"/>
    <remap from="scan" to="$(arg laser_topic)"/>
  </node>
</launch>
\end{lstlisting}
Allerdings wurden hier lediglich Standardparameter verwendet, die aus dem Demonstrationsbeispiel des \lstinline{move_base}{} \cite{WikiMoveBase} Paket stammen. In der Konfigurationsdatei werden wiederum Parameterdateien geladen, wobei die meisten aus dem Paket \lstinline{turtlebot_} \lstinline{navigation}{} \cite{WikiTBNavigation} stammen. Für Versuchszwecke wurden in dieser Konfiguration Parameterdateien für die Kostenkarte und den lokalen Planer durch eigene Implementierungen ersetzt. Dadurch können Änderung an den Parametersätzen vorgenommen werden.

Um die Navigation auszuführen, muss die beiden Launch-Dateien auf Slave und Master ausgeführt werden. Allerdings ist anzumerken, dass nach dem Neustart der Systeme deren Uhrzeiten synchronisiert werden müssen, wofür der Befehl \lstinline{ntpdate}{} herangezogen wird.
\begin{lstlisting}[caption={Anleitung Navigation eines Roboters},captionpos=b]
[S] sudo ntpdate 192.168.0.100
[M] soscore
[S] roslaunch EML_Navigation_Slave  EML_Navigation_Slave.launch
[M] roslaunch EML_Navigation_Master EML_Navigation_Master.launch
\end{lstlisting}