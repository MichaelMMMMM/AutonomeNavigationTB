\section{Systemstruktur}
\pdfcomment{Formulierung, die Erklärung von Master/Slave ist schlecht}
In dieser Arbeit wird die Navigation von maximal zwei Robotern zur selben Zeit betrachtet. Jeder der Roboter ist mit einem PC ausgestattet, die über WLAN mit einem TP-Link-Router verbunden sind, der als Access-Point des Netzwerks fungiert. Der Roboterverbund ist nach einem Master-Slave-Prinzip konzipiert, wobei die Roboter bzw. deren PCs als Slave agieren. Ein weiterer PC ist mit dem Router verbunden und agiert als Master. Der Master-PC ist zusätzlich mit Tastatur, Maus und Bildschirm ausgestattet und stellt somit die Steuereinheit des Systems dar. Daher stammt auch die Bezeichnung Master-PC, da von diesem die Befehlskette bedient wird. Die Slave-PCs, welche sich auf den Robotern befinden, nehmen die Befehle entgegen, arbeiten also als Slaves \pdfcomment{really?}.

Das Netzwerk läuft in dem IP-Adressbereich 192.168.0.X, wobei die Adressen des Master- und der Slave-PCs statisch zugewiesen werden. Der Master-PC ist unter der Adresse 192.168.0.100 erreichbar; Roboter R2 unter der Adresse 192.168.0.102; Roboter R4 unter der Adresse 192.168.0.104. Außerdem ist es möglich, dass sich weitere Rechner mit dem Netzwerk verbinden, wodurch auf sämtliche Daten des ROS-Netzes zugegriffen werden kann. Somit können auch Entwicklungswerkzeuge wie MATLAB verwendet werden, um die Daten abzugreifen und weiterzuverarbeiten.

Das Netzwerk trägt den Namen \lstinline{EML_Turtlebot_NET}{} mit dem Passwort \lstinline{turtlebot}{}. Die statische Konfiguration der IP-Adressen kann im Detail in \cite[S. 23]{Turtleboys} nachgelesen werden.
Auf jedem der Rechner existiert ein Nutzer mit dem Namen \lstinline{turtlebot}{} und Passwort \lstinline{turtlebot}{}, der genutzt werden kann, um eine ssh-Verbindung aufzubauen.